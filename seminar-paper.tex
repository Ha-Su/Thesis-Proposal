%!TEX root = seminar-paper.tex
%%%%%%%%%%%%%%%%%%%%%%%%%%%%%%%%%%%%%%%%%%%%%%%%%%%%%%%%%%%%%%%%%%%%%%%%%%%
\documentclass[sigconf]{acmart}
\settopmatter{printacmref=false}
\renewcommand\footnotetextcopyrightpermission[1]{}

% Copyright
\setcopyright{none}
%\setcopyright{acmcopyright}
%\setcopyright{acmlicensed}
%\setcopyright{rightsretained}
%\setcopyright{usgov}
%\setcopyright{usgovmixed}
%\setcopyright{cagov}
%\setcopyright{cagovmixed}

%Conference
% \acmConference[SEEMOO Seminars Winter 2017/18]{Seminars on Networking, Security, 
% Mobility, and Wireless Communications Winter 2017/18}{February 2018}{Darmstadt, Germany} 
% \acmYear{2018}
% \copyrightyear{2018}
% \acmArticle{4}
% \acmPrice{15.00}

\begin{document}
\title{Your Seminar Topic}

% Comment out the following for you final document
\subtitle{One-page Summary}

\author{Your Name}
\affiliation{%
  \institution{Technische Universit\"at Darmstadt}
}
\email{your.name@stud.tu-darmstadt.de}

% The default list of authors is too long for headers.
\renewcommand{\shortauthors}{Y. Name et al.}


\begin{abstract}
  You may want to answer the following questions with roughly one sentence each.

  \begin{itemize}
  \item What is the topic of your main seminar paper?
  \item What problem does it solve?
  \item Why is that topic/problem important?
  \item What methodologies do the authors apply?
  \item What are the main contributions of the paper?
  \item What are the key findings/results of the paper?
  \end{itemize}
  
  \textbf{Note 1:} You do not have to answer every question above. Consider only those that are relevant to your paper. 

\end{abstract}

\maketitle

\section{Introduction}

Write a short paragraph (5-15 lines) on each of the following tasks (where applicable):
\begin{itemize}
\item Motivate your topic in general.
\item Why is your research question important in that field?
\item Give one practical example.
\item To what existing work is your topic related, what has been done there?
\item What are the (planned) main contributions of your paper? e.g., a
new attacker model, a summary, a comparison, \dots
\item Give an outline of the paper: describe each of your (planned)
sections in one sentence.
\end{itemize}


\section{Background and Related Work}

\section{Relevant Section}

Main content starts here. \nocite{*}


\section{Conclusion}
The conclusion goes here.\\

% Remove this note when submitting your report
\textbf{Note 2:} The first deliverable (one-page summary) consists of \emph{(i)} an abstract, where important questions need to be addressed with short sentences. For instance, you should justify why the topic is important and mention what are the main contributions of the paper. In addition, the one-page summary may also have an \emph{(ii)} introduction section and/or \emph{(iii)} a background section. If you need to create an additional section, feel free to do it. For instance, if you have identified open problems that have not been addressed by the paper, you can create a new section to discuss these aspects (and possibly propose solutions). Finally, in the last section, you should summarize the paper you have read and show your understanding. In the \emph{references section}, include at least 5 relevant works in the topic (that you may want to read later or include in your final paper). \\

\textbf{Note 3:} In Moodle you can find examples of the one-page summary (under the name of \emph{samples}). These examples are meant to be used only as a guidance. Your report will be assessed based on its quality, your understanding of the topic and your capacity of synthesis. By no means, the one-page summary should be an attempt to mimic the structure of the examples. 

\textbf{Note 4:} Be concise and clear with your explanation as the space is limited to a single page. Your one-page summary is expected to cover all the available space in this page. References shall be placed in the next page.

\textbf{Note 5:} For the final paper, you are also expected to use this template. In addition, to the research questions you answered in the one page-summary, you may want to consider the following ones in your final seminar paper.
  \begin{itemize}
  \item What is your research question?
  \item Why are that question and your topic important?
  \item How did you proceed to answer the question?
  \item What (do you think) is the answer to your question?
  \item Give an overall opinion on your topic.
  \item If you have results, describe them.
  \item What is the impact of the answers to your questions?
  \end{itemize}
  
\textbf{Note 6:} Examples for the final paper are also provided in Moodle. Again, these examples are only referential and you are not expected to mimic their structure. 

\textbf{Note 7:} Depending on the type of seminar you are taking, consider the following.

Regular seminar: 7 full pages + unlimited pages for references 

Advanced seminar: 9 full pages + unlimited pages for references 

% References
\newpage
\bibliographystyle{ACM-Reference-Format}
\bibliography{library} 

\end{document}
